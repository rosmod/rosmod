\documentclass[conference]{IEEEtran}

% correct bad hyphenation here
%\hyphenation{op-tical net-works semi-conduc-tor}
\newcommand{\ENABLECOMMENTS}{}
\usepackage{listings}
\usepackage{ifpdf}
\usepackage{wrapfig}
\usepackage{url}
\usepackage{subcaption}
\ifpdf
\usepackage[pdftex]{graphicx}
\else
\usepackage{graphicx}
\fi
\newcommand{\iap}{\textit{DREMS}\ }
%\newcommand{\iapfull}{\textbf{D}istributed \textbf{S}oftware \textbf{P}latform }
\newcommand{\iapfull}{\textbf{D}istributed \textbf{RE}altime \textbf{M}anaged \textbf{S}ystem}


\usepackage{color}

\usepackage{tikz}
\usetikzlibrary{matrix,arrows,circuits.ee,circuits.ee.IEC,shapes.geometric,shapes.misc}
\ifdefined\ENABLECOMMENTS
\newcommand{\comment}[2]{{\texttt{\color{red}#1:#2}}}
\else
\newcommand{\comment}[2]{}
\fi

\begin{document}
%
% paper title
% can use linebreaks \\ within to get better formatting as desired
\title{A Testbed to Simulate and Analyze Resilient Cyber-Physical Systems}


% author names and affiliations
% use a multiple column layout for up to three different
% affiliations
\author{ Pranav Kumar, William Emfinger and Gabor Karsai \\
Institute for Software Integrated Systems,\\ Dept. of EECS, Vanderbilt University,\\
%  1025 16th Ave S, \\
Nashville, TN 37235, USA \\
Email:\{pkumar, emfinger, gabor\}@isis.vanderbilt.edu}


% make the title area
\maketitle

\begin{abstract}
We have created a testbed for the development and testing of Cyber-Physical Systems (CPS) applications.  This testbed incorporates smart network hardware which allows for high-fidelity emulation of the system's network characteristics, and simulation systems which allow for high-fidelity simulation of the CPS, its environment, its sensors, and its actuators.  We discuss the architecture of this testbed, the types of experiments and applications which can be run on the testbed, some of the testbed's limitations, and some extensions to the testbed.  
\end{abstract}
\section{Introduction}

Distributed CPS are hard to develop hardware/software for; because the software is coupled with the hardware and the physical system, software testing and deployment may be difficult.  Many systems require rigorous testing before final deployment, but may not be able to be tested easily in the lab or may not be testable in the real world without first providing the assurances that the tests produce.  These types of systems must be tested for performance assurances, reliability, and fail-safety.  Examples of these systems include UAV/UUV systems, fractionated satellite clusters, and networks of autonomous vehicles, all of which require strict guarantees about not only the performance of the system, but also the reliability of the system.  Because of the need for such strict design-time guarantees, many traditional techniques for software testing cannot be used.  Cloud-based software testing may not accurately reflect the performance of the software, since many of these systems use specialized embedded computers, and furthermore does not provide the capability to easily integrate a system simulation into the software testing loop.  For such systems, a closed-loop simulation testbed is necessary which can fully emulate the deployed system, including the physical characteristics of the nodes, the network characteristics of the systems, and the sensors and actuators used by the systems.

use automotive industry testbed as example, this is cost effective version for other embedded systems developers and researchers


CPS are hard to develop hardware/software for; because the software is coupled with the hardware, software testing and deployment may be difficult.  Many systems require rigorous testing before final deployment, but may not be able to be tested easily in the lab or in the real world without first providing the assurances that the tests produce.  For such systems, a closed-loop simulation testbed is necessary which can fully emulate the deployed system, including the physical characteristics of the nodes, the network characteristics of the systems, and the sensors and actuators which the systems used.  Furthermore, many of these systems use specialized embedded computers which have very different software and hardware support than cloud-based testing infrastructure can provide.  

use automotive industry testbed as example, this is cost effective version for other embedded systems developers and researchers

architectural description:
require good physics integration -> distributed -> time sync -> low jitter -> accurate timing
what kind of tests can you run?
what kind of systems is this good for?

limitations:
can't do all integration testing
usb-ethernet -> limitation for physics possibly -> what scenarios
switches can be limitations depending on applications and such -> to what can they support?
processing limitations -> dependent on testbed -> hardware prototyping boards should be used for relevant hardware
how does the system/testbed scale: switches? openflow?

future work:
test/measure jitter

\input{Related_Research}
\input{Problem_Statement}
\section{The RCPS Testbed}

\subsection{Architecture}
require good physics integration -> distributed -> time sync -> low jitter -> accurate timing
what kind of tests can you run?
what kind of systems is this good for?
\subsubsection{Application Network and Emulation}
What network emulation is; why is it useful; how it is done: Dummynet (like last year's paper) and OpenFlow (like we're trying to do this year)

This network emulates the physical network which the system would use after deployment.  This can be a wired or wireless network or networks with any topology.  Using network emulation techniques the characteristics of the network as it would exist after deployment can be enforced on the testbed's application network traffic to emulate the real network.
\subsubsection{Physics Network}
This network provides the infrastructure necessary to emulate the systems' sensors and actuators, and allows the testbed software to receive sensor data and output actuator commands to the simulation.
\subsubsection{Physics sim and interface}
The physics simulation closes the loop for the testbed, allowing the physical dynamics of the testbed's hardware to be simulated in the environment in which it would be deployed.  Furthermore, an API allows the aspects of the simulation such as sensor data and control commands to be transmitted between the testbed and the simulation.

One such simulator which supports this type of interface and has proven useful for the development and testing of CPS applications is Orbiter Space Flight Simulator.  Orbiter allows for very accurate gravitational dynamics simulation for systems in the solar system.  Using the API provided by the Orbiter add-on OrbConnect, programs can communicate with and control the spacecraft in the simulation.  

Another more recent alternative to Orbiter, Kerbal Space Program (KSP), supports a wider range of physical system simulations, and expedites the process of simulation development and deployment.  Because KSP supports not only gravitational dynamics simulation, but also aerodynamic, rigid-body, and fluid simulation, it can be used to simulation systems which interact with these different domains.  Using the API provided by the KSP add-on KRPC (KSP remote procedure call), programs can control the state of the simulation and communicate with the systems in the simulation.  Whereas simulation development with orbiter required either the use of pre-existing models (e.g. the space-shuttle), or complete 3d modeling and specification of the system components to be simulated, simulation development with KSP is facilitated by the built-in tools KSP has to build aerospace/ground systems from their constituent components.  Because KSP has a large library of built-in components which have been modeled and can be composed together however the designer chooses, the capability to develop new simulations is greatly improved.  
\subsubsection{Actual hardware}
The hardware which runs the testbed application software is composed of 32 BeagleBone Blacks.  These embedded computing boards run Ubuntu Linux on a dual-core 32-bit ARM processor with an embedded GPU.  

\subsection{System Analysis}
\subsubsection{Network Analysis}
\subsubsection{Timing Analysis}

\subsection{Experiments}
\subsubsection{Orbiter}
Good for certain types of system testing but not great for doing tests across multiple domains
\subsubsection{KSP}
Though not as accurate, allows testing between multiple interacting domains.

\subsection{Limitations}
can't do all integration testing
usb-ethernet -> limitation for physics possibly -> what scenarios
switches can be limitations depending on applications and such -> to what can they support?
processing limitations -> dependent on testbed -> hardware prototyping boards should be used for relevant hardware
how does the system/testbed scale: switches? openflow?

\subsection{Future Work/Extensions}
test/measure jitter
\input{Future_Work}
\input{Conclusions}

% Acknowledgement
\section*{Acknowledgment}

This work was supported by ... 

\bibliographystyle{IEEEtran}
%\bibliography{f6}

\end{document}
